\documentclass{article}

\usepackage[utf8]{inputenc}
\usepackage[danish]{babel}
\usepackage{graphicx}

\title{Analysedokument}
\author{Henrik Romby Mikkelsen og Jacob Scherffenberg-Møller}

\begin{document}

\maketitle
\tableofcontents
\pagebreak

\section{Referat}
\emph{Skougruppen - 25 okotober 2012. Møde med Martin Skou}
\hrule

Vi fremviste systemtet til Martin, som ligesom ved vores indledende møde virkede meget entusiastisk, og glad for hjælpe, med udviklingen. 
Han var meget imponeret over systemet, men havde dog nogle input til systemet. 

- Noget værktøj har et officielt navn, men bliver kaldt noget andet, disse kaldenavne, skal der kunne søges efter på værktøjslisten, men de skal ikke fremgå på listen. 

-SMS skal udsendes med kodeordet, nå man bliver oprettet.

-Værktøjslisten skal kunne udskrives.

-Ønske om at søge kasseret værktøj, fandtes i forvejen, endvidere skulle værktøj udover kassering, kunne sættes til "bortkommet", så der kunne holdes styr, på hvor meget værktøj man kasserer, og hvor meget der bliver væk.

-Nyoprettet værktøj, skal kunne tilføjes kommentarer, så man kan holde styr på bilagnumre, dette kan bruges i forbindelse med forsikringssager, kommentarfeltet skal være frivilligt. 

-Service intervallerne var for fine, og der skulle kunne søges på værktøj der har overskredet deres interval.

Vi aftalte at få tilsendt nogle gamle værktøjslister, så vi kan få test systemet med en realistisk mængde data.

Nå vi har rettet de issues, der "opstod" ved mødet skal vi have systemet online, så den lageransvarlige, kan tage systemet i brug. På denne måde, vil vi kunne se om systemet har nogle problemer, der ikke er blevet opdaget, i forbindelse med vores egen test.  
\end{document}
